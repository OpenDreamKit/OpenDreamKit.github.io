
% 2.1 Description of the seminar (3-5 pages, in English):
% - Brief, general introduction to the topic
% - In-depth description of the topic
% - Questions and issues addressed by the seminar; objectives and results
%   expected to be produced by the seminar
% - As applicable: relationship to previous seminars or how the proposed seminar
%   differs from similar seminars. Which new developments and issues are to be
%   addressed? (Tip: Try our Seminar/Events search on the homepage.) Please address
%   also the relation and differences of your proposed seminar to (major)
%   conferences in the same area.

\section{Introduction}

From their earliest days, computers have successfully been used in mathematics
to perform complicated or tedious calculations more reliably, or at all, to make
tables, to prove theorems (famously the four colour theorem), to explore new
theories.

With computers and open source software becoming widely, and cheaply, available
to everyone, the last decades have seen the emergence of open-source tools to
conduct research; the spectrum ranges from special purpose libraries such as
\MPIR or \Linbox, over topical systems such as \Singular, to general purpose
systems that feature a complete programming language and environmen such as
\GAP or \Sage.
These open-source systems are sometimes in direct competition with
(semi-)commerical systems (MAGMA, Mathematica, MatLab...).

Addionally there is a wealth of databases of objects, Atlas, small groups, transitive groups...
\ednote{markus: I might be focusing this too much on pure maths/combinatorics?}

There is also undeniable value in these software packages for teaching and
collaborative work.

The big problem with the multitude of systems with different communities,
developers, and focuses, is that they do not compose well. To have any hope of
using a mathematics software package inside one's own system one has to
understand the conventions and internals of the library sufficiently to write a
bespoke interface, translate data and representations of objects. This is
tedious, time consuming and error-prone.

In an ideal world there would be a single, well defined, standard that is used
to efficiently communicate data between systems.
Such a standard would have to be universal enough to cover enough fields,
efficient enough to not slow down computations significantly, and easy enough to
use to not impose an implementation burden on deveopers.

Since the world rarely is ideal, and one cannot expect every developer of
mathematics software to change it to the needs of this standard, we need to work
towards composability.


A partial success is \Sage, a free general purpose open-source mathematics
software system licensed under the GPL whose mission is to create a
viable free open source alternative to Magma, Maple, Mathematica and
Matlab. It has been developed since 2005 by a growing worldwide community of
about 150 researchers and teachers. It builds on top of many existing
open-source packages, including NumPy, SciPy, matplotlib, Sympy,
Maxima, and the aforementioned ones, all accessible from a
Python-based library containing itself many unique mathematical
features.
Thanks to this, \Sage is regularly used in universities, both for
research and education purposes.

We propose to establish the Math-in-the-Middle approach as described in
\ednote{CICM paper?} based on MMT as the base for exchanging mathematical
objects/meaning/etc; a universal API for mathematics.


\section{Description}

Computer mathematics software is very diverse: in the past every special
interest community developed their software only considering their own
needs, with custom APIs and environments. Some projects are developed in general
purpose programming languages, such as C, Java, or Python, others develop their
own domain specific languages, or custom languages (like GAP or GP).

Contemporary research in computational mathematics often needs access to
multiple diverse specialist libraries (example; groups, number theory, linear algebra...) 

The upshot is that researchers build software by tacking it together with duct
tape and patches, or duplicate effort developing specialist solutions for their
current problems, which comes at cost of fragility, complexity, and doubtful correctness.

But a general framework that connects exists: its formal mathematics, formalised
in the language of logic.

Rabe/Kohlhase et al developed a very general framework (MMT) that can serve as a
translation layer between different logical frameworks.

The vision is to establish this Math-in-the-middle approach as the API for as
many computer mathematics software, without forcing these packages to change
their approach.

Working on WP6 of OpenDreamKit we realised more and more that it is necessary to
communicate and work together in one place

get representatives from fields to com

The seminar will consist of few talks that introduce the core concepts of MMT,
current state of exports of Sage, GAP, Lmfdb MMT interfaces

Lower the entry barrier for using MitM

\section{Questions, Issues, Objectives, Results}



The goal of the seminar is to 
\begin{itemize}
\item Bring together developers of (open source) mathematics software
  components, and logicians and knowledge representation experts.
\item Get an overview of the current trends and developments in open source
  mathematics software.
\item Promote MMT/MathInTheMiddle
\item Train mathematicians in the art of MMT, and get an idea of how MMT will be
  used by developers of domain specific software.
\item Share perspectives and best practices, build a joint vision, and
  seek venues for tighter cooperation.
\item Encourage participants to get involved in the standardisation process and
  to provide MMT interfaces to their software.
\end{itemize}

Some of the upcoming major challenges are:
\begin{itemize}
\item Lower the entry barrier, in particular via \textbf{unified user
    interfaces}, and \textbf{Virtual Research Environments} that
  groups of users can setup to collaborate on data, software,
  computations, or knowledge;
\item Further enable \textbf{computations involving multiple systems},
  as transparently as possible;
\item Keep the development efforts manageable as the size and
  complexity of software systems increase;
\item Train a new generation of users and developers.
\end{itemize}

A key step is to strengthen collaborations \emph{between} the various
communities, in order to:
\begin{itemize}
\item Seek for opportunities for collaboration or outsourcing of
  components to save on development efforts;
\item Share expertise and best practices;
\item Improve cross-systems development workflows.
\end{itemize}


\subsection{Goals of the seminar}

Conference, seminars, etc. often focus on the exchange of ideas. It is
of course one of the goals of our proposed semainar, in particular
brushing a precise state-of-the-art of the above topics. 
However our goals go beyond that: we wold like to develop a common
objective of universality that can unify the communities. In the long run, this requires designing uniform conceptualizations, interchange formats, and interoperability layers.
Therefore, \textbf{the goal of this seminar is to systematically identify the current obstacles to universality, to collect requirements for universality, and to sketch out future solutions}.

\paragraph{Research Questions}
% checking dependency theories
% identification of assumptions: extensionnality, choice, proof
% irrelevance etc
% tools for packaging, maintenaning
Participants will be asked to give short talks that specifically address the following research questions from the perspective of their field:

\begin{itemize}
\item Why are current systems not more interoperable? What design changes are necessary to increase interoperability in the future?
\item What are the current approaches towards interoperability? How successful or promising are they?
\item How can correctness be guaranteed in a distributed setting?
 Should there be a single universal checker (which would be hard to agree on) or many decentral ones (which may preclude interoperability)?
\item How can we design interchange languages that naturally subsume
  existing (and future!) formal systems?
\item Should a logical framework permit the definition of any logical system?
Or do the logics currently implemented have points in common that could
be hard wired into the framework itself?
\item How reasonnable is it to propose a single universal proof format?
Or do we need different formats for different families of
proof systems and a partial interoperability between the formats?
 How should a proof format be evaluated (generality, conciseness,
efficiency of proof-checking, ...)?
\item How should universal proof library be exchanged? Is Web technology
sufficient or do we need specific tools to organize data bases of
proofs?
\item How can we practically and reliably relate individual systems with their representation in an interchange format or a logical framework?
How can two systems agree on the meaning of an exchanged theorem and thus trust each other?
\end{itemize}

\paragraph{Impact on the Research Community}
By challenging participants to address research questions concerning
universality, we do not only raise awareness of the importance of these issues.
We also help identify the key steps towards \emph{proving in the large} and  \emph{universality} of proofs.
This will allow the development of a common objective and framework for interoperable and reusable proof development that is crucial for realizing the full potential of formal mechanizations.

This seminar with the associated Dagstuhl proceedings will provide an overview of the problem, the state of the art of current solutions and the active researchers pursuing them, and the most promising ideas for future solutions.
It will collect and strengthen the small, often-disparate communities that currently work towards universality, e.g., in the very different PxTP (Proof eXchange for Theorem Proving) and LFMTP (dedicated to logical frameworks and meta-languages) workshops.

The seminar will not only allow for cross-fertilization between
 \begin{compactitem}
  \item research on logical frameworks, proof formats, logics, proof engineering, mathematics formalization, and program verification,
  \item foundational research on these topics and application or system-oriented approaches.
 \end{compactitem}
It will also structure and streamline future collaboration, e.g., by kicking off new workshops or large international grant proposals.


